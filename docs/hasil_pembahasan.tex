\documentclass[12pt,a4paper]{article}
\usepackage[utf8]{inputenc}
\usepackage[indonesian]{babel}
\usepackage{listings}
\usepackage{xcolor}
\usepackage{geometry}
\usepackage{hyperref}
\usepackage{booktabs}
\usepackage{longtable}
\usepackage{caption}

\geometry{margin=2.5cm}

\definecolor{codebg}{RGB}{245,245,245}
\definecolor{codegreen}{RGB}{0,128,0}
\definecolor{codegray}{RGB}{128,128,128}
\definecolor{codeblue}{RGB}{0,0,180}
\definecolor{codepurple}{RGB}{128,0,128}

\lstdefinestyle{javastyle}{
    backgroundcolor=\color{codebg},
    basicstyle=\ttfamily\footnotesize,
    keywordstyle=\color{codeblue}\bfseries,
    stringstyle=\color{codegreen},
    commentstyle=\color{codegray}\itshape,
    numberstyle=\tiny\color{codegray},
    numbers=left,
    numbersep=8pt,
    breaklines=true,
    frame=single,
    rulecolor=\color{codegray},
    tabsize=4,
    showstringspaces=false,
    language=Java,
    morekeywords={var, record, sealed, permits},
    literate={á}{{\'a}}1 {é}{{\'e}}1 {í}{{\'i}}1 {ó}{{\'o}}1 {ú}{{\'u}}1
}

\lstset{style=javastyle}

\title{\textbf{Hasil dan Pembahasan\\Unit Testing pada Kelas Wallet\\menggunakan JUnit 5}}
\author{PPPL -- Tugas 1}
\date{}

\begin{document}

\maketitle
\tableofcontents
\newpage

% ============================================================
\section{Pendahuluan}
% ============================================================

Dokumen ini membahas implementasi kelas \texttt{Wallet} beserta pengujian unitnya
(\texttt{WalletTest}) yang ditulis dalam bahasa Java dengan framework \textbf{JUnit 5}.
Proyek ini menggunakan \textbf{Maven} sebagai build tool dan Java 21 sebagai versi compiler.

Tujuan utama proyek adalah mendemonstrasikan praktik \textit{unit testing} yang baik,
mencakup:
\begin{itemize}
    \item Pengujian tiap method secara independen.
    \item Pengujian kasus normal (\textit{happy path}) dan kasus tepi (\textit{edge case}).
    \item Pemanfaatan \textit{lifecycle annotations} JUnit 5
          (\texttt{@BeforeAll}, \texttt{@BeforeEach}, \texttt{@AfterEach}, \texttt{@AfterAll}).
\end{itemize}

% ============================================================
\section{Struktur Proyek}
% ============================================================

\begin{verbatim}
src/
├── main/java/org/example/
│   └── Wallet.java          <- kode produksi
└── test/java/org/example/
    └── WalletTest.java       <- kode pengujian
\end{verbatim}

% ============================================================
\section{Pembahasan Kelas \texttt{Wallet.java}}
% ============================================================

Kelas \texttt{Wallet} merepresentasikan sebuah dompet digital sederhana yang dapat
menyimpan identitas pemilik, kartu-kartu, dan uang tunai dalam bentuk lembaran.

% ------------------------------------------------------------
\subsection{Atribut Kelas}
% ------------------------------------------------------------

\begin{lstlisting}
private String owner;
private final List<String> cards = new ArrayList<>();
private final List<Integer> cashList = new ArrayList<>();
\end{lstlisting}

\begin{itemize}
    \item \texttt{owner} -- Menyimpan nama pemilik dompet.
          Bertipe \texttt{String} dan bernilai \texttt{null} secara default
          sampai di-\textit{set} secara eksplisit.
    \item \texttt{cards} -- Daftar kartu (misal KTP, SIM, ATM) yang disimpan di dompet.
          Dideklarasikan \texttt{final} agar referensi list tidak bisa diganti,
          namun isi list tetap bisa dimodifikasi.
    \item \texttt{cashList} -- Daftar lembaran uang tunai.
          Setiap elemen merepresentasikan satu lembar uang dengan nominal tertentu
          (misal 10000, 20000, 50000).
\end{itemize}

% ------------------------------------------------------------
\subsection{Method \texttt{setOwner} dan \texttt{getOwner}}
% ------------------------------------------------------------

\begin{lstlisting}
public void setOwner(String owner) {
    this.owner = owner;
}

public String getOwner() {
    return owner;
}
\end{lstlisting}

\textbf{Fungsi:} Mengatur dan mengambil nama pemilik dompet.

\textbf{Logika:}
Method ini merupakan \textit{getter} dan \textit{setter} standar.
Nilai awal \texttt{owner} adalah \texttt{null}; setelah \texttt{setOwner("Budi")}
dipanggil, \texttt{getOwner()} akan mengembalikan \texttt{"Budi"}.

\textbf{Alasan implementasi:}
Enkapsulasi -- atribut \texttt{owner} bersifat \texttt{private},
sehingga akses dari luar kelas hanya melalui method ini.

% ------------------------------------------------------------
\subsection{Method \texttt{addCard}}
% ------------------------------------------------------------

\begin{lstlisting}
public void addCard(String card) {
    if (card != null && !card.isEmpty()) {
        cards.add(card);
    }
}
\end{lstlisting}

\textbf{Fungsi:} Menambahkan kartu ke dalam dompet.

\textbf{Logika:}
Sebelum menambahkan, method memvalidasi bahwa parameter \texttt{card}
tidak \texttt{null} dan tidak kosong (\texttt{""}). Jika validasi gagal,
kartu \textbf{tidak} ditambahkan dan tidak ada \textit{exception} yang dilemparkan.
Pendekatan ini disebut \textit{silent rejection}.

\textbf{Alasan implementasi:}
Validasi input mencegah data tidak valid masuk ke dalam dompet.
Tanpa validasi, list \texttt{cards} bisa berisi \texttt{null} atau string kosong
yang akan menyulitkan pemrosesan di kemudian hari.

% ------------------------------------------------------------
\subsection{Method \texttt{takeCard}}
% ------------------------------------------------------------

\begin{lstlisting}
public String takeCard(String card) {
    if (cards.remove(card)) {
        return card;
    }
    return null;
}
\end{lstlisting}

\textbf{Fungsi:} Mengambil (menghapus) kartu dari dompet berdasarkan nama kartu.

\textbf{Logika:}
\texttt{cards.remove(card)} mengembalikan \texttt{true} jika elemen berhasil dihapus.
Jika kartu ditemukan dan dihapus, method mengembalikan nama kartu tersebut.
Jika tidak ditemukan, method mengembalikan \texttt{null}.

\textbf{Alasan implementasi:}
Mengembalikan objek yang diambil memungkinkan pemanggil mengetahui
apakah operasi berhasil (\textit{non-null}) atau gagal (\texttt{null}).

% ------------------------------------------------------------
\subsection{Method \texttt{getCards}}
% ------------------------------------------------------------

\begin{lstlisting}
public List<String> getCards() {
    return new ArrayList<>(cards);
}
\end{lstlisting}

\textbf{Fungsi:} Mengembalikan daftar kartu dalam dompet.

\textbf{Logika:}
Method ini mengembalikan \textbf{salinan} (\textit{defensive copy}) dari list internal,
bukan referensi langsung. Artinya, jika pemanggil memodifikasi list yang dikembalikan
(misalnya menambah atau menghapus elemen), list asli di dalam objek \texttt{Wallet}
\textbf{tidak terpengaruh}.

\textbf{Alasan implementasi:}
\textit{Defensive copy} adalah praktik penting dalam OOP untuk menjaga
\textit{encapsulation}. Tanpa ini, kode eksternal bisa memanipulasi state internal
objek secara langsung, melanggar prinsip \textit{information hiding}.

% ------------------------------------------------------------
\subsection{Method \texttt{addCash}}
% ------------------------------------------------------------

\begin{lstlisting}
public void addCash(int amount) {
    if (amount > 0) {
        cashList.add(amount);
    }
}
\end{lstlisting}

\textbf{Fungsi:} Menambahkan lembaran uang tunai ke dompet.

\textbf{Logika:}
Hanya menerima nilai positif (\texttt{amount > 0}).
Nilai nol dan negatif ditolak secara diam-diam (\textit{silent rejection}).
Setiap pemanggilan menambahkan satu elemen ke \texttt{cashList}.

\textbf{Alasan implementasi:}
Uang tunai tidak bisa bernilai nol atau negatif dalam dunia nyata,
sehingga validasi ini mencerminkan aturan bisnis (\textit{business rule}) yang logis.

% ------------------------------------------------------------
\subsection{Method \texttt{takeCash}}
% ------------------------------------------------------------

\begin{lstlisting}
public Integer takeCash(int amount) {
    if (cashList.remove(Integer.valueOf(amount))) {
        return amount;
    }
    return null;
}
\end{lstlisting}

\textbf{Fungsi:} Mengambil (menghapus) satu lembaran uang dengan nominal tertentu.

\textbf{Logika:}
\texttt{Integer.valueOf(amount)} diperlukan agar Java memanggil
\texttt{List.remove(Object)} (hapus berdasarkan nilai), bukan
\texttt{List.remove(int)} (hapus berdasarkan indeks).
Jika nominal ditemukan, lembaran pertama dengan nominal tersebut dihapus dan
nilainya dikembalikan. Jika tidak ditemukan, dikembalikan \texttt{null}.

\textbf{Alasan implementasi:}
Perbedaan antara \texttt{remove(int)} dan \texttt{remove(Object)} pada \texttt{List}
adalah \textit{pitfall} klasik di Java. Penggunaan \texttt{Integer.valueOf()} secara
eksplisit memastikan perilaku yang benar.

% ------------------------------------------------------------
\subsection{Method \texttt{getCashList}}
% ------------------------------------------------------------

\begin{lstlisting}
public List<Integer> getCashList() {
    return new ArrayList<>(cashList);
}
\end{lstlisting}

\textbf{Fungsi:} Mengembalikan daftar lembaran uang di dompet.

\textbf{Logika:}
Sama seperti \texttt{getCards()}, method ini menerapkan \textit{defensive copy}
untuk melindungi state internal dari modifikasi eksternal.

% ------------------------------------------------------------
\subsection{Method \texttt{getTotalCash}}
% ------------------------------------------------------------

\begin{lstlisting}
public int getTotalCash() {
    return cashList.stream().mapToInt(Integer::intValue).sum();
}
\end{lstlisting}

\textbf{Fungsi:} Menghitung total uang tunai di dompet.

\textbf{Logika:}
Menggunakan \textbf{Java Stream API}:
\begin{enumerate}
    \item \texttt{cashList.stream()} -- membuat stream dari list.
    \item \texttt{mapToInt(Integer::intValue)} -- mengonversi setiap \texttt{Integer}
          menjadi \texttt{int} primitif (menghasilkan \texttt{IntStream}).
    \item \texttt{sum()} -- menjumlahkan semua elemen.
\end{enumerate}
Jika list kosong, \texttt{sum()} mengembalikan \texttt{0}.

\textbf{Alasan implementasi:}
Stream API memberikan kode yang ringkas dan deklaratif dibandingkan loop manual.

% ============================================================
\section{Pembahasan Kelas \texttt{WalletTest.java}}
% ============================================================

Kelas ini berisi 22 method pengujian yang dikelompokkan berdasarkan method
yang diuji.

% ------------------------------------------------------------
\subsection{Lifecycle Annotations}
% ------------------------------------------------------------

\begin{lstlisting}
@BeforeAll
static void initAll() {
    testCount = 0;
    System.out.println("=== Mulai test WalletTest ===");
}

@BeforeEach
void setUp() {
    wallet = new Wallet();
    testCount++;
    System.out.println("Test #" + testCount + " dimulai");
}

@AfterEach
void tearDown() {
    System.out.println("Test #" + testCount + " selesai");
    wallet = null;
}

@AfterAll
static void tearDownAll() {
    System.out.println("=== Semua " + testCount + " test selesai ===");
}
\end{lstlisting}

\begin{itemize}
    \item \texttt{@BeforeAll} -- Dijalankan \textbf{sekali} sebelum seluruh test dimulai.
          Menginisialisasi counter dan mencetak pesan awal.
    \item \texttt{@BeforeEach} -- Dijalankan \textbf{sebelum setiap} test method.
          Membuat objek \texttt{Wallet} baru agar setiap test berjalan dengan
          state bersih (\textit{test isolation}).
    \item \texttt{@AfterEach} -- Dijalankan \textbf{setelah setiap} test method.
          Mencetak pesan dan meng-\texttt{null}-kan referensi \texttt{wallet}.
    \item \texttt{@AfterAll} -- Dijalankan \textbf{sekali} setelah seluruh test selesai.
          Mencetak ringkasan jumlah test.
\end{itemize}

\textbf{Alasan implementasi:}
\textit{Test isolation} memastikan setiap test independen satu sama lain.
Jika satu test memodifikasi dompet, test berikutnya tidak terpengaruh
karena objek baru selalu dibuat di \texttt{@BeforeEach}.

% ------------------------------------------------------------
\subsection{Pengujian \texttt{setOwner} / \texttt{getOwner}}
% ------------------------------------------------------------

\begin{longtable}{p{4.5cm} p{8.5cm}}
\toprule
\textbf{Test Method} & \textbf{Penjelasan} \\
\midrule
\endhead
\texttt{testSetOwner} &
Memverifikasi bahwa setelah \texttt{setOwner("Budi")} dipanggil,
\texttt{getOwner()} mengembalikan \texttt{"Budi"}.
Menggunakan \texttt{assertEquals}. \\
\midrule
\texttt{testOwnerAwalnyaNull} &
Memverifikasi bahwa owner bernilai \texttt{null} sebelum di-set.
Menggunakan \texttt{assertNull}. \\
\midrule
\texttt{testOwnerTidakNullSetelahDiSet} &
Memverifikasi bahwa setelah di-set, owner tidak lagi \texttt{null}.
Menggunakan \texttt{assertNotNull}. \\
\bottomrule
\end{longtable}

% ------------------------------------------------------------
\subsection{Pengujian \texttt{addCard}}
% ------------------------------------------------------------

\begin{longtable}{p{4.5cm} p{8.5cm}}
\toprule
\textbf{Test Method} & \textbf{Penjelasan} \\
\midrule
\endhead
\texttt{testAddCard} &
Menambahkan \texttt{"KTP"}, lalu memverifikasi ukuran list = 1
dan list mengandung \texttt{"KTP"}.
Menggunakan \texttt{assertEquals} dan \texttt{assertTrue}. \\
\midrule
\texttt{testAddBanyakCard} &
Menambahkan 3 kartu dan memverifikasi ukuran list = 3.
Menguji akumulasi data. \\
\midrule
\texttt{testAddCardNull} &
Menambahkan \texttt{null} dan memverifikasi list tetap kosong.
Menguji validasi input terhadap \texttt{null}. \\
\midrule
\texttt{testAddCardKosong} &
Menambahkan string kosong \texttt{""} dan memverifikasi list tetap kosong.
Menguji validasi input terhadap string kosong. \\
\bottomrule
\end{longtable}

% ------------------------------------------------------------
\subsection{Pengujian \texttt{takeCard}}
% ------------------------------------------------------------

\begin{longtable}{p{5cm} p{8cm}}
\toprule
\textbf{Test Method} & \textbf{Penjelasan} \\
\midrule
\endhead
\texttt{testTakeCard} &
Menambahkan 2 kartu, mengambil \texttt{"KTP"}, lalu memverifikasi
return value = \texttt{"KTP"} dan \texttt{"KTP"} sudah tidak ada di list. \\
\midrule
\texttt{testTakeCardYangTidakAda} &
Mengambil kartu yang tidak ada di dompet.
Memverifikasi return value = \texttt{null}. \\
\midrule
\texttt{testTakeCardDariWalletKosong} &
Mengambil kartu dari dompet yang benar-benar kosong.
Memverifikasi return value = \texttt{null} tanpa \textit{exception}. \\
\bottomrule
\end{longtable}

% ------------------------------------------------------------
\subsection{Pengujian \texttt{addCash}}
% ------------------------------------------------------------

\begin{longtable}{p{4.5cm} p{8.5cm}}
\toprule
\textbf{Test Method} & \textbf{Penjelasan} \\
\midrule
\endhead
\texttt{testAddCash} &
Menambahkan 50000, memverifikasi ukuran list = 1
dan list mengandung 50000. \\
\midrule
\texttt{testAddBanyakCash} &
Menambahkan 3 lembaran uang, memverifikasi ukuran list = 3. \\
\midrule
\texttt{testAddCashNol} &
Menambahkan nilai 0, memverifikasi list tetap kosong.
Menguji \textit{boundary condition} \texttt{amount > 0}. \\
\midrule
\texttt{testAddCashNegatif} &
Menambahkan nilai negatif, memverifikasi list tetap kosong.
Menguji validasi terhadap nilai negatif. \\
\bottomrule
\end{longtable}

% ------------------------------------------------------------
\subsection{Pengujian \texttt{takeCash}}
% ------------------------------------------------------------

\begin{longtable}{p{5.5cm} p{7.5cm}}
\toprule
\textbf{Test Method} & \textbf{Penjelasan} \\
\midrule
\endhead
\texttt{testTakeCash} &
Menambahkan 2 lembaran, mengambil 50000.
Memverifikasi return = 50000 dan 50000 tidak ada lagi di list. \\
\midrule
\texttt{testTakeCashYangTidakAda} &
Mengambil nominal yang tidak ada.
Memverifikasi return = \texttt{null}. \\
\midrule
\texttt{testTakeCashDariWalletKosong} &
Mengambil uang dari dompet kosong.
Memverifikasi return = \texttt{null}. \\
\bottomrule
\end{longtable}

% ------------------------------------------------------------
\subsection{Pengujian \texttt{getTotalCash}}
% ------------------------------------------------------------

\begin{longtable}{p{5.5cm} p{7.5cm}}
\toprule
\textbf{Test Method} & \textbf{Penjelasan} \\
\midrule
\endhead
\texttt{testGetTotalCash} &
Menambahkan 10000 + 20000 + 50000, memverifikasi total = 80000.
Menguji penjumlahan dengan Stream API. \\
\midrule
\texttt{testTotalCashWalletKosong} &
Memverifikasi total = 0 pada dompet kosong.
Menguji kondisi list kosong agar tidak menghasilkan error. \\
\midrule
\texttt{testTotalCashSetelahAmbilUang} &
Menambahkan 50000 + 20000, mengambil 20000,
memverifikasi total = 50000.
Menguji konsistensi total setelah operasi \texttt{takeCash}. \\
\bottomrule
\end{longtable}

% ------------------------------------------------------------
\subsection{Pengujian Skenario Gabungan}
% ------------------------------------------------------------

\begin{longtable}{p{5.5cm} p{7.5cm}}
\toprule
\textbf{Test Method} & \textbf{Penjelasan} \\
\midrule
\endhead
\texttt{testWalletLengkap} &
Mengisi dompet secara lengkap (owner, kartu, uang),
lalu memverifikasi semua data secara bersamaan.
Menguji integrasi antar fitur. \\
\midrule
\texttt{testGetCardsReturnSalinan} &
Menambahkan 1 kartu, lalu memodifikasi list yang dikembalikan
\texttt{getCards()}. Memverifikasi list internal tetap berukuran 1.
Menguji \textit{defensive copy}. \\
\midrule
\texttt{testSameCardReference} &
Menambahkan kartu \texttt{"KTP"}, mengambilnya kembali,
lalu memverifikasi bahwa referensi objek sama (\texttt{assertSame}).
Menguji \textit{identity equality}. \\
\midrule
\texttt{testBedaInstanceWallet} &
Membuat 2 objek \texttt{Wallet} berbeda dengan owner berbeda.
Memverifikasi bahwa kedua referensi owner tidak sama
(\texttt{assertNotSame}).
Menguji independensi antar instance. \\
\bottomrule
\end{longtable}

% ============================================================
\section{Hasil Pengujian}
% ============================================================

Seluruh 22 test method berhasil dijalankan tanpa kegagalan.
Ringkasan hasil:

\begin{center}
\begin{tabular}{l c c}
\toprule
\textbf{Kategori Pengujian} & \textbf{Jumlah Test} & \textbf{Status} \\
\midrule
\texttt{setOwner / getOwner}  & 3  & \textcolor{green!60!black}{PASSED} \\
\texttt{addCard}               & 4  & \textcolor{green!60!black}{PASSED} \\
\texttt{takeCard}              & 3  & \textcolor{green!60!black}{PASSED} \\
\texttt{addCash}               & 4  & \textcolor{green!60!black}{PASSED} \\
\texttt{takeCash}              & 3  & \textcolor{green!60!black}{PASSED} \\
\texttt{getTotalCash}          & 3  & \textcolor{green!60!black}{PASSED} \\
Skenario Gabungan              & 4  & \textcolor{green!60!black}{PASSED} \\
\midrule
\textbf{Total}                 & \textbf{22} & \textcolor{green!60!black}{\textbf{ALL PASSED}} \\
\bottomrule
\end{tabular}
\end{center}

% ============================================================
\section{Analisis dan Kesimpulan}
% ============================================================

\subsection{Teknik yang Digunakan}

\begin{enumerate}
    \item \textbf{Validasi Input} --
          Method \texttt{addCard} dan \texttt{addCash} memvalidasi parameter
          sebelum memodifikasi state. Ini mencegah data tidak valid masuk
          ke dalam objek.

    \item \textbf{Defensive Copy} --
          Method \texttt{getCards()} dan \texttt{getCashList()} mengembalikan
          salinan list, bukan referensi langsung. Ini menjaga enkapsulasi
          dan mencegah modifikasi state secara tidak sengaja dari luar kelas.

    \item \textbf{Stream API} --
          Method \texttt{getTotalCash()} menggunakan Stream API untuk
          perhitungan agregat yang ringkas dan \textit{readable}.

    \item \textbf{Test Isolation} --
          Setiap test method mendapatkan objek \texttt{Wallet} baru melalui
          \texttt{@BeforeEach}, sehingga tidak ada ketergantungan antar test.

    \item \textbf{Assertion yang Beragam} --
          Pengujian memanfaatkan berbagai jenis assertion JUnit 5:
          \texttt{assertEquals}, \texttt{assertNull}, \texttt{assertNotNull},
          \texttt{assertTrue}, \texttt{assertFalse}, \texttt{assertSame},
          dan \texttt{assertNotSame}.
\end{enumerate}

\subsection{Kesimpulan}

Kelas \texttt{Wallet} telah diimplementasikan dengan menerapkan prinsip
\textit{encapsulation}, validasi input, dan \textit{defensive copy}.
Seluruh 22 unit test berhasil \textit{passed}, membuktikan bahwa:

\begin{itemize}
    \item Setiap method bekerja sesuai spesifikasi pada kondisi normal.
    \item \textit{Edge case} seperti input \texttt{null}, string kosong,
          nilai nol, dan nilai negatif ditangani dengan benar.
    \item \textit{Defensive copy} mencegah kebocoran state internal.
    \item Operasi pada dompet kosong tidak menyebabkan \textit{exception}.
\end{itemize}

\end{document}
